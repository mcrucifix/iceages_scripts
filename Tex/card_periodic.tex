\begin{figure}
\begin{center}
\includegraphics[scale=0.75]{Figures/Pdf/card_periodic}
\end{center}
\vskip-2.5em
\caption
{
Bifurcation diagram obtained by counting the number of points on the pullback section in the van der Pol oscillator ($\alpha=30, \quad \beta=0.7$) and  $F(t)=\gamma \sin( 2\pi / P_1 t + \phi_{P1})$. The two x-axes indicate $\tau$ and the ratio between the natural system period and the forcing, respectively.  One observes the synchronisation regimes corresponding to 1:1, 2:1, 3:1, 4:1 and 5:1, respectively (gray, blue, red, green, yellow) and, intertwinned, higher order synchronisations including $2:3$, $2:5$, $3:2$ etc.  White areas are weak or no synchronisation. Graph constructed using $t_\mathrm{back}=-10$Ma (see Figure \ref{fig:convergence} and text for meaning and implications.)
}
\label{fig:card_periodic}
\end{figure}
